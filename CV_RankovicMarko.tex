%!TEX TS-program = xelatex
\documentclass[]{rankovic-cv}
\usepackage{afterpage}
\usepackage{hyperref}
\usepackage{color}
\usepackage{xcolor}
\usepackage[utf8x,utf8]{inputenc} % make weird characters work
\usepackage[serbian]{babel}

\hypersetup{
    pdftitle={CV_RankovicMarko},
    pdfauthor={Marko Ranković},
    pdfsubject={Curriculum Vitae},
    colorlinks=false,       % no lik border color
   allbordercolors=white    % white border color for all
}
\addbibresource{bibliography.bib}
\RequirePackage{xcolor}
\definecolor{pblue}{HTML}{0395DE}

\begin{document}
\header{Marko}{Ranković}
      {Bachelor of Computer Science}
      

\fcolorbox{white}{gray}{\parbox{\dimexpr\textwidth-2\fboxsep-2\fboxrule}{%
.....
}}


\begin{aside}
  \section{Address}
    Arčibalda Rajsa 18
    Belgrade, Serbia
    ~
  \section{Tel \& Skype}
    +381 69 637467
    sudzum1
    ~
  \section{Mail}
    \href{mailto:marko.rankovic@outlook.com}{\textbf{marko.rankovic@}\\outlook.com}
    ~
  \section{Web \& Git}
    \href{https://www.linkedin.com/in/marko-rankovic-055034b0}{Marko Ranković LinkedIn}
    \href{https://github.com/marko-rankovic}{Marko Ranković GitHub}
    ~
  \section{Programming}
    \textbf{C/C++}\includegraphics[scale=0.40]{img/5stars.png}
    \textbf{SQL}\includegraphics[scale=0.40]{img/4stars.png}
    \textbf{Java}\includegraphics[scale=0.40]{img/2stars.png}
    \textbf{.NET}\includegraphics[scale=0.40]{img/2stars.png}
    \textbf{Python}\includegraphics[scale=0.40]{img/3stars.png}
    ~
  \section{OS Preference}
    \textbf{GNU/Linux}\includegraphics[scale=0.40]{img/5stars.png}
    \textbf{Unix}\includegraphics[scale=0.40]{img/3stars.png}
    \textbf{MacOS}\includegraphics[scale=0.40]{img/4stars.png}
    \textbf{Windows}\includegraphics[scale=0.40]{img/3stars.png}
    ~
  \section{Personal Skills}
    \textbf{Teamplay}\includegraphics[scale=0.40]{img/5stars.png}
    \textbf{Commun.}\includegraphics[scale=0.40]{img/5stars.png}
    \textbf{Curious}\includegraphics[scale=0.40]{img/5stars.png}
    \textbf{Initiative}\includegraphics[scale=0.40]{img/4stars.png}
    \textbf{Organized}\includegraphics[scale=0.40]{img/4stars.png}
    ~
\end{aside}

\section{Experience}
\begin{entrylist}
  \entry
    {02/15 - Now}
    {Embedded Software Engineer}
    {RT-RK Institute, Belgrade, Serbia}
    {Design and development of embedded software     for Android-based devices working on MIPS processors.\\}
  \entry
    {11/13 - 02/15}
    {Intern (Embedded Software Development)}
    {RT-RK Institute, Belgrade, Serbia}
    {Design and development of Browser-based HLS Media Players in PNACL.\\}
\end{entrylist}
~
\section{Education}
\begin{entrylist}
  \entry
    {2015}
    {Master's Degree in Computer Science}
    {Faculty of Mathematics, Belgrade,Serbia}
    {Main subjects: Informational Systems, Data Mining, Artificial Inteligence, Networks.}
  \entry
    {2011 - 2015}
    {Bachelor's Degree in Computer Science}
    {Faculty of Mathematics, Belgrade, Serbia}
    {Main subjects: Mathematics, Programming, Databases, Operating Systems, Hardware Architecture.}
  \entry
    {2007 - 2011}
    {Vuk Karadžić Diplomma}
    {High school "Mladost", Petrovac na Mlavi, Serbia}
    {Secondary School.\\
    Main subjects: Matematics, Physics, Computer Science.}
\end{entrylist}
~
\section{Certifications}
\begin{entrylist}
  \entry
    {10/2015}
    {Git / Gerrit}
    {RT-RK Institute, Belgrade, Serbia}
    {\emph{Fundamentals of Git and Gerrit}}
  \entry
    {07/2015}
    {Android system programming}
    {RT-RK Institute, Belgrade, Serbia}
    {\emph{Introduction to Android system architecture and programming}}
  \entry
    {12/2014}
    {Digital TV}
    {RT-RK Institute, Belgrade, Serbia}
    {\emph{Introduction to DTV architecture and programming for STB}}
  \entry
    {04/2014}
    {Introduction to .NET }
    {University of Belgrade, Serbia}
    {\emph{Basics of CS.NET}}
  \entry
    {02/2014}
    {Nvidia CUDA Programming}
    {University of Belgrade, Serbia}
    {\emph{Introduction to programming CUDA kernels for Nvidia graphic cards}}
\end{entrylist}
~
\section{Projects}
\begin{entrylist}
  \entry
    {06/2014}
    {Pangaea}
    {University of Belgrade, Serbia}
    {\emph{Turn-based, client-server game similar to RISK, made in Qt 5}}
  \entry
    {06/2014}
    {JDBC school project}
    {University of Belgrade, Serbia}
    {\emph{Database project for Student Administration office made using JDBC and Swing}}
  \entry
    {11/2013}
    {Unnamed}
    {Private project, Belgrade, Serbia}
    {\emph{Unity racing game made for Android devices.}}  
\end{entrylist}
~
\end{document}
 